% -----------------------------------------------------------------
% 				   Probablility and Statistics
% -----------------------------------------------------------------
\begin{tcolorbox}[colback=red!5!white,colframe=red!75!black,title=Probablility and Statistics]
Random Variables and Probability
\begin{equation*}
P(A|B) \cdot  P(B) = P(B|A) \cdot  P(A)
\end{equation*}
\begin{equation*}
P(A|B) = \frac { P(A,B) }{ P(B) }  \rightarrow  \frac { P(B,A) \cdot  P(A) }{ P(B) } 
\end{equation*}
\begin{equation*}
P(X \in  [a,b]) = \int _{ a }^{ b }{  p_{ X }(x) dx } 
\end{equation*}
Mean
\begin{equation*}
\mu_X = \mathbb{ E }\{ f(x)\}  := \int _{ -\infty  }^{ \infty  }{  f(x) \cdot  p_{ X }(x) dx } 
\end{equation*}
\begin{equation*}
\mathbb{ E }\{ a + bX\}  := a + b\mathbb{ E }\{ X\} 
\end{equation*}
Variance
\begin{equation*}
\sigma _{ X }^{ 2 } := \mathbb{ E }\{ (X-\mu _{ X })^{ 2 }\}  = \mathbb{ E }\{ X^{ 2 }\} -\mu _{ X }^{ 2 }
\end{equation*}
\begin{equation*}
std dev \, \sigma _{ X } = \sqrt { variance \, \sigma _{ X }^{ 2 } } 
\end{equation*}
\end{tcolorbox}

\begin{tcolorbox}[colback=red!5!white,colframe=red!75!black,title=Distributions]
Uniform distribution:
\begin{equation*}
{ P_{ y }(x) } = \left\{ \begin{matrix} \frac { 1 }{ b-a }  &  \quad if\quad x  \in  [a,b] \\ 0 & else \end{matrix} \right.
\end{equation*}
Normal (Gaussian) distribution:
\begin{equation*}
p(x) = \frac{1}{\sqrt{2\pi \sigma^2}}\cdot exp(-\frac{(x-\mu)^2}{2\sigma^2})
\end{equation*}
\begin{equation*}
X \sim \mathcal{N}(\mu, \sigma^2)
\end{equation*}
Multidimensional Normal Distribution:
\begin{equation*}
p(x)=\frac { 1 }{ \sqrt { (2\pi )^{ n }\cdot det(\Sigma ) }  } \cdot exp(-\frac { 1 }{ 2 } \, \cdot \, { (x-\mu ) }^{ T }\, \cdot \, \Sigma ^{ -1 }\, \cdot \, (x-\mu )\, )
\end{equation*}
Weibull distribuation: 
\begin{equation*}
F(x) \, = \, 1 - e^{-(\lambda \cdot x)^k}
\end{equation*}
Laplace distribuation:
\begin{equation*}
f(x|\mu,b) = \frac{1}{2b} \cdot exp (-\frac{|x-\mu|}{b})
\end{equation*}
\end{tcolorbox}

\begin{tcolorbox}[colback=red!5!white,colframe=red!75!black,title=Useful Definitions]
Covariance and Correlaton:
\begin{equation*}
\sigma (Y,Z) := \mathbb{E} {(Y-\mu_Y)(Z-\mu_Z)} = 
\end{equation*}
\begin{equation*}
= \int_{-\infty}^{\infty}{\int_{-\infty}^{\infty}{(y-\mu_Y)(z - \mu_Z)\cdot p_{Y,Z} (y,z)\,  dy \, dz} }
\end{equation*}
Covariance Matrix:
\begin{equation*}
\Sigma_x = cov(X) = \mathbb{E}\{XX^T\} - \mu_x \mu^T_x
\end{equation*}
Multidimensional Random Variables:
\begin{equation*}
\mathbb{E}{f(X)} = \int_{\mathbb{R}^n}{f(x)p_X(x) d^n x}
\end{equation*}
\begin{equation*}
cov(X) = \mathbb{E} \{(X-\mu_X)(X-\mu_X)^T\} 
\end{equation*}
\begin{equation*}
cov(X) = \mathbb{E} \{XX^T\} - \mu_X \mu_X^T
\end{equation*}
\begin{equation*}
cov(Y) = \Sigma_y = A \Sigma_x A^T \quad for \quad y = A \cdot x
\end{equation*}
\begin{equation*}
\mathbb{E }\{AX\} =A \cdot \mathbb{ E }\{X\}
\end{equation*}
Rules for variance:
\begin{equation*}
var(X+Y) = var(X)+var(Y)+2 \cdot cov(X,Y)
\end{equation*}
\begin{equation*}
var(aX) = {a}^{2} \cdot var(X)
\end{equation*}
Verschiebesatz:
\begin{equation*}
var(X)={ { \mathbb{E}((X-\mathbb{E}(X)) }^{ 2 } })=\mathbb{E}({ X }^{ 2 })-{ (\mathbb{E}(X)) }^{ 2 }
\end{equation*}

unit Variance is variance = 1

Statistical estimators:\\
\textbf{Biased- and unbiasedness} $\rightarrow$ an estimator ${\hat{\theta}}_{N}$ is called unbiased iff $\mathbb{E}\{ {\hat{\theta}}_{N} ({y}_{N})\} = \theta_0$, where ${\theta}_{0}$ is the true value of a parameter. Otherwise, is called biased.

\textbf{Asymptotic Unbiasedness} $\rightarrow$ An estimator ${\hat{\theta}}_{N}$ is called asymptotically unbiased iff 
$
\lim\limits_{n \to \infty} \mathbb{E}\{ {\hat{\theta}}_{N} ({y}_{N}) \} = \theta_0
$

\textbf{Consistency} $\rightarrow$ An estimator ${\hat{\theta}}_{N} ({y}_{N})$ is called consistent if, for any $ \epsilon > 0$, the probability $
P( {\hat{\theta}}_{N} ({y}_{N}) \in [\theta_0 - \epsilon, \theta_0 + \epsilon])
$ tends to one as $N \rightarrow \infty$.
\end{tcolorbox}

\begin{tcolorbox}[colback=green!5!white,colframe=green!75!black,title=Unconstrainded Optimization]
	\begin{description}
		\item[Theorem 1:]{(First Order Necessary Conditions)\\
			If $x^* \in D$ is local minimizer of $f : D \rightarrow \mathbb{R}$ and $f \in C^1$ then
			$\triangledown f (x^*) = 0$
			Definition (Stationary Point) A point $\bar{x}$ with $\triangledown f(\bar{x}) = 0$ is called a stationary point of f.
			}
		\item[Theorem 2:] {(Second Order Necessary Conditions)\\
			If $x^* \in D$ is local minimizer of $f : D \rightarrow R$ and $f \in C^2$ then
			$\triangledown^2 f(x^*) \succeq 0$
			}
		\item[Theorem 3:] {(Second Order Sufficient Conditions and Stability under Perturbations)\\
			Assume that $f : D \rightarrow R$ is $C^2.$ If $x^* \in D$ is a stationary point and
			$ \triangledown^2 f(x^*) \succ 0$
			then $x^*$ is a strict local minimizer of f. In addition, this minimizer is locally unique and is stable against small perturbations of f, i.e. there exists a constant C such that for sufficiently small $p \in \mathbb{R}^n$ holds\\
			\begin{equation*}
			\parallel{x^* - arg\, \underset{x}{min}  (f(x) + p^T x)}\parallel \,\, \leq \, C\parallel{p}\parallel
			\end{equation*}
			}
	\end{description}
\end{tcolorbox}