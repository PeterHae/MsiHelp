% -----------------------------------------------------------------
%	 Non parametric and frequency domain identification methods
% -----------------------------------------------------------------
\begin{tcolorbox}[colback=brown!5!white,colframe=brown!75!black,title=\textbf{Fourier Transformation}]
	\textbf{FT:}\begin{align*}
	F\{F\}(\omega) = \int_{-\infty}^{\infty} f(t) e^{-j\omega t}dt
	\end{align*}
	\textbf{iFT:}\begin{align*}
	f(t) = F^{-1}\{F\}(t) = \frac{1}{2\pi}\int_{-\infty}^{\infty} F(\omega) e^{-j\omega t}d\omega
	\end{align*}
	\textbf{DFT:}\begin{align*}
	U(m) := \sum_{k = 0}^{N-1} u(t)e^{-j\frac{2\pi m k}{N}}
	\end{align*}
	\textbf{iDFT:}\begin{align*}
	u(n) := \sum_{k = 0}^{N-1} U(k)e^{j\frac{2\pi k n}{N}}
	\end{align*}
	
\todo[inline]{Wirklich notwendig? Eher nict aber das drüber :)}
\textbf{How to compute FT?} By DFT, which solves the problem of finite time and discrete values.

\textbf{Can we use an input with many frequencies to get many FRF (Frequency Response Function) values in a single experiment?} So far only frequency sweeping (high comp. times due to repetition for each frequency). We should use multisines!
\end{tcolorbox}

\begin{tcolorbox}[colback=brown!5!white,colframe=brown!75!black,title=\textbf{Useful frequency things}]
$\omega = 2\pi f = \frac{2\pi}{T}$ \quad $f_s > 2 f_{max}$ \quad $T = N \Delta{t} = \frac{N}{f_s} $\quad 
\end{tcolorbox}

\begin{tcolorbox}[colback=brown!5!white,colframe=brown!75!black,title=\textbf{Aliasing and Leakage Errors}]
\begin{description}
	\item[Aliasing Error:] Due to sampling of continous signal to discrete signal. Avoid with Nyquist Theoreme:
	\begin{equation*}
	f_{Nyquist} = \frac{1}{2\Delta t} [\SI{}{\hertz}] \quad or \quad \omega_{Nyquist} = \frac{2\pi}{2\Delta t} [\SI{}{\radian \per \second}]
	\end{equation*}
	
	\item[Leackage Error:] Due to windowing.
	\begin{equation*}
	\omega_{base} := \frac{2\pi}{N \cdot \Delta t} = \frac{2\pi}{T} \rightarrow \omega = m \frac{2\pi }{N \cdot \Delta t}
	\end{equation*}
\end{description}
\end{tcolorbox}

\begin{tcolorbox}[colback=brown!5!white,colframe=brown!75!black,title=\textbf{Crest Factor = Scheitelfaktor}]
\begin{flalign*}
	\text{Crest Factor } =\frac{u_{max}}{u_{rms}} \quad \text{with}: u_{rms} &:= \sqrt{\frac{1}{T} \int_{0}^{T} u(t)^2 \, dt} &\\
	\quad \text{and} \quad u_{max} &:=\underset{t \in [0,T]}{max} |u(t)| &
\end{flalign*}
\end{tcolorbox}

\begin{tcolorbox}[colback=brown!5!white,colframe=brown!75!black,title=\textbf{Optimising Multisine for optimal crest factor}]
\begin{description}
	\item[Frequency:] Choose frequencies in logarithmic manner as multiples of the base frequency. $\quad \omega_{k+1}/\omega_k \approx 1.05$

	\item[Phase:] To prevent high peaks (Crest Factor) in the Signal, the phases of the different frequencies are modulated accordingly. (Positive interference)
\end{description}
\end{tcolorbox}

\begin{tcolorbox}[colback=brown!5!white,colframe=brown!75!black,title=\textbf{Multisine Identification Implementation procedure}]
\begin{description}
	\item[Window Length:] Integer multiple of sampling time: $T = N \cdot \delta t$

	\item[Harmonics of base frequency:] Are contained in multisine \\ $\omega_{base} = \frac{2\pi}{T}$

	\item[Highest contained Frequency:] Is \textbf{half} of Nyquist frequency: $\omega_{Nyquist} = \frac{2\pi}{4\Delta T}$

	\item[Experiment and Analysis:] (Step 2): Insert Multisine periodically. Drop first Periods (till transients died out). Record M Periods, each with N samples, of input and output data. Average all the M periods and make the DFT (or vice versa). Finally build transfer function: $ \hat{G}_{{j\omega}_{k}} = \frac{\hat Y(k(p))}{\hat U (k(p))}$
\end{description}
\end{tcolorbox}

\begin{tcolorbox}[colback=brown!5!white,colframe=brown!75!black,title=\textbf{Nonparametric and Frequency Domain Identification Models}]
\textbf{Impulse response and transfer function:}
\begin{align*}
	y(t) &= \int_{0}^{\infty} g(\tau) u(t-\tau) \, \delta t \\
	Y(s) &= G(s)\cdot U(s) \\
	G(s) &= \int_{0}^{\infty} \epsilon^{-st} g(t) \, dt 
\end{align*}
\textbf{Bode diagram from frequency sweeps:}
\begin{align*}
	u(t) &= A \cdot sin(\omega \cdot t),\quad y(t) = \lVert G (j\cdot \omega )\rVert A \cdot sin(\omega \cdot t + \alpha)
\end{align*}
\end{tcolorbox}

\begin{tcolorbox}[colback=brown!5!white,colframe=brown!75!black,title=\textbf{Bode Diagramm}]
	Magnitude = Amplitude $|G(j\omega)|$\\
	Phase $arg \, G(j\omega)$
	\todo[inline]{Hier sollten wir glaub noch bisschen was rein machen.}
\end{tcolorbox}


\begin{tcolorbox}[colback=violet!5!white,colframe=violet!75!black,title=\textbf{Recursive Least Squares}]
\begin{flalign*}
	& \textbf{New Inverse Covariance: } Q_K = Q_{k-1} + \phi_K \phi_K^T &
\end{flalign*}
\end{tcolorbox}

\begin{tcolorbox}[colback=violet!5!white,colframe=violet!75!black,title=\textbf{Kalman Filter}]
\textbf{Valid for Discrete and Linear!}
\begin{flalign*}
	& \text{If recursive least squares: } x_{ k+1 }\, =\, A_{ k }\cdot x_{ k }& \\
	& x_{k+1} = A_k \cdot x_k + \omega_k \quad \text{and} \quad y_k = C_k \cdot x_k + v_k&
\end{flalign*}

\textbf{Steps of Kalman Filter}
\begin{description}
	\item[\small 1 Prediction] 
		\begin{flalign*}
		& \hat x_{[k | k-1]} = A_{k-1} \cdot \hat x_{[k-1 | k-1]} & \\ 
		& P_{[k | k-1]} = A_{k-1} \cdot P_{[k-1 | k-1]} \cdot A_{k-1}^{T} \cdot W_{k-1} & \\
		& \text{If RLS, without: } W_{k-1} &
		\end{flalign*}
		
	\item[\small 2 Innovation update] 
		\begin{flalign*}
		& P_{[k | k]} = (P_{[k | k-1]}^{-1} + C_{k}^{T} \cdot V^{-1} \cdot C_k)^{-1} & \\ 
		& \hat x_{[k | k]} = \hat x_{[k | k-1]} + P_{[k | k]} \cdot C_{k}^{T} \cdot V^{-1} \cdot (y_k - C_k \cdot \hat x_{[k | k-1]}) &
		\end{flalign*}
\end{description}
\todo[inline]{Ein Beispiel einfuegen.}
\end{tcolorbox}

